\section{Munkatapasztalat}
\cventry{2024--}{Data scienctist}{Quantco}{}{Zürich}{A Quantco új-generációs biztosítás-árazási platformjának fejlesztése.}
\cventry{2023}{Data science gyakornok}{Quantco}{}{Zürich}{A Quantco GLM modellek becslésére szolgáló open-source Python csomagjának (\href{https://github.com/Quantco/glum}{glum}) fejlesztése. Formulainterfész, rugalmas Wald-tesztek és egyéb hasznos funkciók hozzáadása. A csomag háttérrendszerét szolgáló csomag (\href{https://github.com/Quantco/tabmat}{tabmat}) refaktorálása egységesebb interfész érdekében. Agzakornokság alatt végzett munka képezi a \href{https://github.com/Quantco/glum/releases/tag/v3.0.0}{glum v3} és a \href{https://github.com/Quantco/tabmat/releases/tag/4.0.0}{tabmat v4} kiadások alapját.}
\cventry{2018--2019}{Elemző}{Magyar Nemzeti Bank}{Pénzügyi Rendszer Elemzése Igazgatóság}{}{Egy IFRS9-nek megfelelő modell megtervezése és implementálása a bankrendszeri top-down stressz-teszthez. Hitelfelvételi korlátok becslése vállalati szinten egy disequilibrium modell használatával. Az EU-támogatások pénzügyi korlátokra gyakorolt hatásának vizsgálata.}
\cventry{2016--2018}{Junior elemző}{Magyar Nemzeti Bank}{Pénzügyi Rendszer Elemzése Igazgatóság}{}{Hatásvizsgálat készítése a gazdaságfejlesztési célú EU-támogatásokra. A stresszteszthez szükséges szatellitmodellek becslése és futtatása.}
\cventry{2014, 2015 nyár}{Gyakornok}{Magyar Nemzeti Bank}{PénzügPwra
yi Rendszer Elemzése Igazgatóság}{}{A top-down stresszteszthez használt hitelbedőlési modell kiegészítése a nemfizető hitelek gyógyulására vonatkozó becsléssel.}
